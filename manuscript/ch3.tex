\chapter{Postprocesando archivos ODB}

\section{Procesando información básica}

\subsection{Partes}

Las partes o geometrías son las entidades básicas en Abaqus, y a partir las cuales se construye el ensamble 
o modelo. Las partes pueden ser bidimensionales o tridimensionales, deformables o rígidas, dependiendo 
de las características del cuerpo físico a representar.\\

Para leer información acerca de las partes incluidas en un archivo ODB, debemos primeramente leer el archivo 
y enseguida acceder al diccionario \texttt{parts}.\\

En el siguiente código se imprimen en consola todas las partes que contiene el archivo \textit{ejemplo.odb}, 
y adicionalmente se imprime el tipo de la parte en cuestión (Deformable, Analítica,...).

\begin{python}
from odbAccess import openOdb

dbpath = "ejemplo.odb"
odb = openOdb(path=dbpath)
for _name,_prt in odb.parts.items():
	print _name, _prt.type
\end{python}

Pero claro, siempre que sea posible es mejor escribir código que pueda ser reutilizado, en forma de funciones y/o clases 
que puedan almacenarse en módulos y posteriormente importarse en un script donde sean utilizadas.\\

En el siguiente código se define una función {\tt get\_parts} que básicamente lee la información de las partes que 
componen el archivo de salida, devolviendo una lista de tuplas con los nombres y tipos de las partes.

\begin{python}
from odbAccess import *

def get_parts(dbpath):
	odb = openOdb(path=dbpath)
	_parts = []
	for _name,_prt in odb.parts.items():
		_type = _prt.type
		_parts.append((_name,_type))
	return _parts
\end{python}


\subsection{Secciones}

\subsection{Materiales}

\subsection{Pasos de carga}

\subsection{Interacciones}

\subsection{Instancias}
